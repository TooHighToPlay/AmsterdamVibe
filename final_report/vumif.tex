\documentclass[12pt, a4paper, lithuanian]{article}

\usepackage[utf8x]{inputenc}

\usepackage{VUMIF}
\usepackage{listings}

% "define" Scala
\lstdefinelanguage{scala}{
  morekeywords={abstract,case,catch,class,def,%
    do,else,extends,false,final,finally,%
    for,if,implicit,import,match,mixin,%
    new,null,object,override,package,%
    private,protected,requires,return,sealed,%
    super,this,throw,trait,true,try,%
    type,val,var,while,with,yield},
  otherkeywords={=>,<-,<\%,<:,>:,\#,@},
  sensitive=true,
  morecomment=[l]{//},
  morecomment=[n]{/*}{*/},
  morestring=[b]",
  morestring=[b]',
  morestring=[b]""
}

\usepackage{color}
\definecolor{dkgreen}{rgb}{0,0.6,0}
\definecolor{gray}{rgb}{0.5,0.5,0.5}
\definecolor{mauve}{rgb}{0.58,0,0.82}


% Default settings for code listings
\lstset{frame=tb,
  language=scala,
  aboveskip=3mm,
  belowskip=3mm,
  showstringspaces=false,
  columns=flexible,
  basicstyle={\small\ttfamily},
  numbers=none,
  numberstyle=\tiny\color{gray},
  %keywordstyle=\color{blue},
  %commentstyle=\color{dkgreen},
  %stringstyle=\color{mauve},
  frame=single,
  breaklines=true,
  breakatwhitespace=true
  tabsize=3
}

% \usepackage[mathcsdepttitle]{VUMIF} % --- matematinės informatikos katedros
%     titulinio puslapio formatavimas

% Titulinio puslapio reikalai
\vumifpaper{Project report}
\title{Amsterdam Vibe}
\engtitle{Intelligent Web Applications course final project}
\author{
    \\
    Žilvinas Kučinskas \\
    Student number: 2547940 \\
    E-mail: zil.kucinskas@gmail.com
}

\supervisor{
    Mihnea Dobrescu-Balaur \\
    Student number: 2549278 \\
    E-mail: mihnea@linux.com
}
\reviewer{
  Arthur-Ervin Avramiea \\
  Student number: 2517642 \\
  E-mail: a.e.avramiea@student.vu.nl
}
\date{Amsterdam \\ 2014}

\begin{document}

\maketitle

\tableofcontents

\section{Introduction}

This is comprehensive report of Intelligent Web Applications course final group project.

\subsection{Requirements}

There was the following requirements for the project:

\begin{itemize}
    \item Use an RDF store.

    \item Use semantic Web reasoning in your RDF store to generate new information.

    \item Integrate at least three data sources.

    \item Present the integrated information in cool, interesting and innovative ways.

    \item Interact with at least one remote SPARQL endpoint (In addition to your local RDF store).

    \item Interact with at least one non RDF Web service.

    \item Write a report about it.

\end{itemize}


\subsection{Code}

All code can be found in the following public Github repository:

\begin{itemize}

  \item https://github.com/TooHighToPlay/AmsterdamVibe

  \item or www.amsterdamvibe.nl

\end{itemize}

\subsection{Link to working application}

Working example of the application can be found on the following link:

\begin{itemize}

  \item amsterdamvibe.herokuapp.com

\end{itemize}

\section{Report}

\subsection{Questions to cover}

\begin{itemize}
  
    \item the goal of the application (what does it aim to do, and why is this useful?).

    \item the datasets and services used by the application

    \item the functionality of the application (what things does the application do, what is a typical workflow)

    \item the inferencing used by the application (it helps if you give a concrete example).

    \item any other considerations you had during the design and implementation (what worked, what didn't work, what motivated your decision to go for a particular solution)

    \item any future plans you may have (what would you like to add if you had the time?)

\end{itemize}

\subsection{Idea}

  Amsterdam is famous not only for it's architecture, history or beautiful sights, but also for it's vibrant nightlife. It has more than 4 million tourists coming over during the year, it also has a lot of youth people from all over the world living here. Amsterdam can offer a lot of electronic music events for such diverse mix of people. 
  Because people in Amsterdam is so proactive, they have a lot of activities and tend to plan parties in advance. Students study hard, work hard and tend to party hard. But when there are so many events, it's easy to miss interesting ones. Youth tend to search events via Facebook, going through club pages, looking at invitations, etc.. But maybe there are an easier, more convenient way of planning parties? 

\subsection{Goal}

  Amsterdam Vibe project was proposed to solve this issue, and to develop an application to help people know the latest and comprehensive information about events going in the town and to help them decide where to go.
  Here are the following main goals:

\begin{itemize}

  \item Compare with other existing online applications, providing information about music events, and evaluate them to provide best user experience with Amsterdam Vibe application.

  \item Provide comprehensive information about events and music artists.

  \item Suggest places by providing personalization in Amsterdam Vibe application.

  \item Make it simple and easy to use (that means less user interaction events, for example mouse clicks, to reach relevant information comparing to analyzed alternatives)

\end{itemize}

\subsection{Comparison}

  As with writing of the report, there were 4 main online sources found, which provide relevant information. Here the typical workflow of navigating and searching information will be provided.

  First is of course Facebook. It's known as the world's most famous and popular social network. Typical workflow of searching events is typing your favorite night club name, opening it's Facebook page and looking at timeline or Events tab, selecting desired event and looking at information provided there. It usually lacks samples of music, user must remember club names and it requires a lot of typing and clicking from the user perspective. It's easier when your friends invite you to an event, but then you cannot compare it with other events happening that day.

  

\subsection{Functionality}

\subsection{Datasets and services}

\subsection{Inferencing}

\subsection{Challenges}

\subsection{Future work}


\end{document}
