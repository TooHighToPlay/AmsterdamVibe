\documentclass[12pt, a4paper, lithuanian]{article}

\usepackage[utf8x]{inputenc}

\usepackage{VUMIF}
\usepackage{listings}

% "define" Scala
\lstdefinelanguage{scala}{
  morekeywords={abstract,case,catch,class,def,%
    do,else,extends,false,final,finally,%
    for,if,implicit,import,match,mixin,%
    new,null,object,override,package,%
    private,protected,requires,return,sealed,%
    super,this,throw,trait,true,try,%
    type,val,var,while,with,yield},
  otherkeywords={=>,<-,<\%,<:,>:,\#,@},
  sensitive=true,
  morecomment=[l]{//},
  morecomment=[n]{/*}{*/},
  morestring=[b]",
  morestring=[b]',
  morestring=[b]""
}

\usepackage{color}
\definecolor{dkgreen}{rgb}{0,0.6,0}
\definecolor{gray}{rgb}{0.5,0.5,0.5}
\definecolor{mauve}{rgb}{0.58,0,0.82}


% Default settings for code listings
\lstset{frame=tb,
  language=scala,
  aboveskip=3mm,
  belowskip=3mm,
  showstringspaces=false,
  columns=flexible,
  basicstyle={\small\ttfamily},
  numbers=none,
  numberstyle=\tiny\color{gray},
  %keywordstyle=\color{blue},
  %commentstyle=\color{dkgreen},
  %stringstyle=\color{mauve},
  frame=single,
  breaklines=true,
  breakatwhitespace=true
  tabsize=3
}

% \usepackage[mathcsdepttitle]{VUMIF} % --- matematinės informatikos katedros
%     titulinio puslapio formatavimas

% Titulinio puslapio reikalai
\vumifpaper{Project report}
\title{Amsterdam Vibe}
\engtitle{Intelligent Web Applications course final project}
\author{
    \\
    Žilvinas Kučinskas \\
    Student number: 2547940 \\
    E-mail: zil.kucinskas@gmail.com
}

\supervisor{
    Mihnea Dobrescu-Balaur \\
    Student number: 2549278 \\
    E-mail: mihnea@linux.com
}
\reviewer{
  Arthur-Ervin Avramiea \\
  Student number: 2517642 \\
  E-mail: a.e.avramiea@student.vu.nl
}
\date{Amsterdam \\ 2014}

\begin{document}

\maketitle

\tableofcontents

\section{Introduction}

This is comprehensive report of Intelligent Web Applications course final group project.

\subsection{Requirements}

There was the following requirements for the project:

\begin{itemize}
    \item Use an RDF store.

    \item Use semantic Web reasoning in your RDF store to generate new information.

    \item Integrate at least three data sources.

    \item Present the integrated information in cool, interesting and innovative ways.

    \item Interact with at least one remote SPARQL endpoint (In addition to your local RDF store).

    \item Interact with at least one non RDF Web service.

    \item Write a report about it.

\end{itemize}


\subsection{Code}

All code can be found in the following public Github repository:

\begin{itemize}

  \item https://github.com/TooHighToPlay/AmsterdamVibe

\end{itemize}

\subsection{Link to working application}

Working example of the application can be found on the following link:

\begin{itemize}

  \item amsterdamvibe.herokuapp.com

  \item or www.amsterdamvibe.nl

\end{itemize}

\section{Report}

\subsection{Questions to cover}

\begin{itemize}
  
    \item the goal of the application (what does it aim to do, and why is this useful?).

    \item the datasets and services used by the application

    \item the functionality of the application (what things does the application do, what is a typical workflow)

    \item the inferencing used by the application (it helps if you give a concrete example).

    \item any other considerations you had during the design and implementation (what worked, what didn't work, what motivated your decision to go for a particular solution)

    \item any future plans you may have (what would you like to add if you had the time?)

\end{itemize}

\subsection{Idea}

  Amsterdam is famous not only for it's architecture, history or beautiful sights, but also for it's vibrant nightlife. It has more than 4 million tourists coming over during the year, it also has a lot of youth people from all over the world living here. Amsterdam can offer a lot of electronic music events for such diverse mix of people. 
  Because people in Amsterdam is so proactive, they have a lot of activities and tend to plan parties in advance. Students study hard, work hard and tend to party hard. But when there are so many events, it's easy to miss interesting ones. Youth tend to search events via Facebook, going through club pages, looking at invitations, etc.. But maybe there are an easier, more convenient way of planning parties? 

\subsection{Goal}

  Amsterdam Vibe project was proposed to solve this issue, and to develop an application to help people know the latest and comprehensive information about events going in the town and to help them decide where to go.
  Here are the following main goals:

\begin{itemize}

  \item Compare with other existing online applications, providing information about music events, and evaluate them to provide best user experience with Amsterdam Vibe application.

  \item Provide comprehensive information about events and music artists.

  \item Suggest places by providing personalization in Amsterdam Vibe application.

  \item Make it simple and easy to use (that means less user interaction events, for example mouse clicks, to reach relevant information comparing to analyzed alternatives).

\end{itemize}

\subsection{Clubs and events}

  By analyzing reviews and information found on the web, articles and based on personal and local oponions, the list of most famous clubs was made. It contains clubs, such as: Air, Bitterzoet, Canvas, Chicago Social Club, Club Nl, Club Up, Escape, Jimmy Woo, Melkweg, Odeon, OT301, Pacific Parc, Paradiso, Studio 80, Sugar Factory, Supperclub and Trouw. The list went shorther, because where was no data in Facebook pages about the upcoming events for the following clubs: Canvas, Club Nl and Jimmy Woo. All in all, Amsterdam Vibe application, as for the first version, is intended to cover events of these 14 clubs in Amsterdam.

\subsection{Comparison}

  As with writing of the report, there were 4 main online sources found, which provide relevant information. The typical workflow of navigating and searching information will be provided.

  First is of course Facebook. It's known as the world's most famous and popular social network. Typical workflow of searching events is typing your favorite night club name, opening it's Facebook page and looking at timeline or Events tab, selecting desired event and looking at information provided there. It usually lacks samples of music, user must remember club names and it requires a lot of typing and clicking from the user perspective. It's easier when your friends invite you to an event, but then you cannot compare it with other events happening that day.

  http://www.partyearth.com/amsterdam/ - It promotes not only club events, but also bars and concerts. It does a great job by providing similar venues and user reviews, but it lacks information about music artists.

  http://thedjlist.com/ - It has a lot information about electronic music overall. You can enter the city and the site provides events in the typed location. It lacks information about artists. It provides information about friends going to event via Facebook, map of the event and integration with Facebook to manage invitations to events. 

  http://partyflock.nl/ - website, started at 2001, it's a Dutch viral community for people interested in electronic music. It has a lot of information about music artists and events, but site user interface is not so friendly. It's a bit hard to navigate the site and it's overloaded with information.

\subsection{Functionality}

  Amsterdam Vibe application tries to incorporate best features found in these sites and add new ones to provide the best experience for the user.

  Design of Amsterdam Vibe main page reminds Pinterest board by splitting events and arranging them depending on screen resolution. The first page of the application, provides the login button. User can log in into the system to get some interesting benefits mentioned in the next paragraph.

  Main application screen provides two types of events:

\begin{itemize}

  \item Suggested events - when user logs into Facebook via Amsterdam Vibe application, it suggests events to the user by analyzing their likes and events, which user attended in the past. It tries to find specific genres of music, which would probably user would like.

  \item Top Events - top events of best clubs in Amsterdam sorted out by date, so the nearest event comes first in the agenda.

\end{itemize}

  Main screen provides event picture, concrete club, which publish this event, time and genre about it.

  When user clicks on specific event, comprehensive information about event shows on the screen. It provides user with description of the event, price, people attending (From Facebook), music genres of the event, music artist list with embedded music samples, start time of the event, cover picture, place, and location visualised on the map.

\subsection{Datasets and services}

  The following datasets and RESTful services were used.

\begin{itemize}

  \item Facebook API - retrieving events of selected clubs pages, getting description, location, date, price, link, cover photo. Also getting past events and user likes to personalize his content.

  \item SoundCloud API - retrieving music samples by music artirst and embedding into application.

  \item Open Street Map API - visualizing location of place, where event is held.

  \item DBPedia Spotlight service - extracting music artist entities from description text.

  \item DBPedia data set - getting relevant information about artists.

  \item Sesame data store - saving all information acquired and inferencing rules.

\end{itemize}

\subsection{Inferencing}

Inferencing was used to suggest users events depending on their past likes and events, in which they participated. Here are some rules, which was used in Amsterdam Vibe application:

\begin{itemize}
  
    \item IF user at event X AND event X has genre G than user may like X.

    \item IF user at event X AND event X has artist A and NOT(user dislike A)  than user may like A.

    \item IF user (likes or may like) genre G1 and genre G2 related to genre G1 than user may like G2.

    \item IF user (likes or may like) genre G AND artist A has genre G AND NOT (user dislike A) than user may like A.

    \item IF future event F has artist A and user may like artist A than user suggested event F.

    \item IF future event F has artist A and user likes artist A user suggested event F.

    \item IF future event F has genre G AND user may like genre G than user suggested event F.
  
    \item IF future event F has genre G AND user likes genre G than user suggested event F.

    \item IF suggested event F AND F has artist A AND user likes A then reason for suggestion A.

    \item IF suggested event F AND F has genre G AND user likes A then reason for suggestion G.

    \item If an event is suggested, and its genre is related to the one of a past event at which the user was present,  than reason for suggestion is (past event, past event genre).

\end{itemize}

\subsection{Challenges}

  There was some challenges making this application and some cool features was dismissed because lack of information and time. First of all, Amsterdam Vibe application was intended to distinguish events, focuse to local people from more internationally oriented events. But there was obstacle, because Facebook API does not let to take information about user, specifically where he lives and where he is from, if retriever is not friend with that person. Web Scrapper was written with Watir/Selenium to login and fetch information about users, but it's too slow for the first release of Amsterdam Vibe.
  Secondly, there was not enough information about music artists sometimes. Idea to fetch that information from site mentioned earlier, partyflock.nl, was proposed. Proof of concept was tested by writing a web scrapper, but that was also dismissed due to lack of time.

\subsection{Future work}

  For the next release of Amsterdam Vibe, it would be awesome to implement features, which was dismissed due to lack of time. It should be possible to maintain information locally about artists and people attending to events, and casually run scrapper jobs to get the latest information. So Amsterdam Vibe application could provide even more comprehensive data about events.

  In addition, it could be possible to analyze countries, from where people come to the event. It is possible that it would factor some people's decision of attending to specific events.

\end{document}
