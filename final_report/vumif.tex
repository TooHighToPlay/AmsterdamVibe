\documentclass[12pt, a4paper, lithuanian]{article}

\usepackage[utf8x]{inputenc}
\def\LTfontencoding{L7x}
\usepackage[\LTfontencoding]{fontenc}
\usepackage[lithuanian]{babel}

\usepackage{VUMIF}
\usepackage{listings}

% "define" Scala
\lstdefinelanguage{scala}{
  morekeywords={abstract,case,catch,class,def,%
    do,else,extends,false,final,finally,%
    for,if,implicit,import,match,mixin,%
    new,null,object,override,package,%
    private,protected,requires,return,sealed,%
    super,this,throw,trait,true,try,%
    type,val,var,while,with,yield},
  otherkeywords={=>,<-,<\%,<:,>:,\#,@},
  sensitive=true,
  morecomment=[l]{//},
  morecomment=[n]{/*}{*/},
  morestring=[b]",
  morestring=[b]',
  morestring=[b]""
}

\usepackage{color}
\definecolor{dkgreen}{rgb}{0,0.6,0}
\definecolor{gray}{rgb}{0.5,0.5,0.5}
\definecolor{mauve}{rgb}{0.58,0,0.82}


% Default settings for code listings
\lstset{frame=tb,
  language=scala,
  aboveskip=3mm,
  belowskip=3mm,
  showstringspaces=false,
  columns=flexible,
  basicstyle={\small\ttfamily},
  numbers=none,
  numberstyle=\tiny\color{gray},
  %keywordstyle=\color{blue},
  %commentstyle=\color{dkgreen},
  %stringstyle=\color{mauve},
  frame=single,
  breaklines=true,
  breakatwhitespace=true
  tabsize=3
}

% \usepackage[mathcsdepttitle]{VUMIF} % --- matematinės informatikos katedros
%     titulinio puslapio formatavimas

% Titulinio puslapio reikalai
\vumifpaper{Project report}
\title{Amsterdam Vibe}
\engtitle{Intelligent Web Applications course final project}
\author{
    \\
    Žilvinas Kučinskas \\
    Student number: 2547940 \\
    E-mail: zil.kucinskas@gmail.com
}

\supervisor{
    Mihnea Dobrescu-Balaur \\
    Student number: 2549278 \\
    E-mail: mihnea@linux.com
}
\reviewer{
  Arthur-Ervin Avramiea \\
  Student number: 2517642 \\
  E-mail: a.e.avramiea@student.vu.nl
}
\date{Amsterdam \\ 2014}

\begin{document}

\maketitle

\tableofcontents

\section{Introduction}

This is comprehensive report of Intelligent Web Applications course final group project.

\subsection{Requirements}

There was the following requirements for the project:

\begin{itemize}
    \item Use an RDF store.

    \item Use semantic Web reasoning in your RDF store to generate new information.

    \item Integrate at least three data sources.

    \item Present the integrated information in cool, interesting and innovative ways.

    \item Interact with at least one remote SPARQL endpoint (In addition to your local RDF store).

    \item Interact with at least one non RDF Web service.

    \item Write a report about it.

\end{itemize}


\subsection{Code}

All code can be found in the following public Github repository:

\begin{itemize}

  \item https://github.com/TooHighToPlay/AmsterdamVibe

  \item or www.amsterdamvibe.nl

\end{itemize}

\subsection{Link to working application}

Working example of the application can be found on the following link:

\begin{itemize}

  \item amsterdamvibe.herokuapp.com

\end{itemize}

\section{Report}

\subsection{Questions to cover}

\begin{itemize}
  
    \item the goal of the application (what does it aim to do, and why is this useful?).

    \item the datasets and services used by the application

    \item the functionality of the application (what things does the application do, what is a typical workflow)

    \item the inferencing used by the application (it helps if you give a concrete example).

    \item any other considerations you had during the design and implementation (what worked, what didn't work, what motivated your decision to go for a particular solution)

    \item any future plans you may have (what would you like to add if you had the time?)

\end{itemize}

\subsection{Idea}

\subsection{Goal}

\subsection{Functionality}

\subsection{Datasets and services}

\subsection{Inferencing}

\subsection{Challenges}

\subsection{Future work}


\end{document}
